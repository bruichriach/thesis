\documentclass[10pt,a4paper]{article}
\usepackage[latin1]{inputenc}
\usepackage{amsmath}
\usepackage{amsfonts}
\usepackage{amssymb}
\usepackage{graphicx}
\title{Thesis Plan}
\begin{document}

\maketitle


\section{Introduction}

\subsection{Arctic Ocean}

Basically the Arctic part of my CofS

\subsubsection{Physical Ocean}

Description of the constituents of the Arctic
and its Dynamics. Topography,
vertical stratification, water masses, circulation.

\subsubsection{Numerical Models}

AOMIP and attempts to have predictive models of
the Ocean. Prominent problems and successes.

\subsection{Mean-Eddy Interaction Theory}

Overview of Residual Theory from GM in 1990 and the
stuff in the late 80's through to M and M 2013.

\subsubsection{GM}

\subsubsection{PV Closure}

I.e. Greatbach

\subsubsection{TWA Equations}

I.e. Young 2012

\subsubsection{Eliason-Palm Tensor}

I.e. M and M 2013

\subsection{Shallow Water Equations}

My own development of the shallow water equations

\subsubsection{Shallow Water Theory}

Rational for using shallow water equations and
description. I.e. Arctic water masses, 
strong stratification.

Note the use of Rigid Lid approximation in
simulations.

\subsubsection{Model Discretisation}

C-grid, AB time stepping, Preconditioned CG solver. 

\subsubsection{Forcing and Dissipation}

The form of wind stress and bottom friction in
shallow water models (e.g. $\tau/h$ and 
$-r\left|\boldsymbol{u}\right|^{2}\boldsymbol{u}/h$). Viscosity:
laplacian, biharmonic as well as Smagorinsky.
Linear, quadratic and n-polynomial bottom friction.

\subsubsection{Eddy Stresses}

Thickness weighted mean of the shallow water
equation. Form of the momentum and pv eddy stresses.
Thickness weighted PV vs PV derived from thickness weighted equations.

Time averaged diagnostics. Explain that this requires steady state dynamics. Need configurations
with a unique mean steady state.

\subsection{Thesis Outline}

Outline of hypotheses.

\section{Zero-Mean Wind Forced Models}

Model where the mean forcing is quasi-zero. 
Model is forced by a residual wind stress which 
invokes a turbulent cascade. Hence Eddy-Mean
interaction is entirely from the eddy to the mean.

\subsection{Mean Eddy Balances}

Break down of the tendency terms for the mean
dynamics

\subsection{Bounding of E-P Tensor}

A look at how the E-P Tensor is characterised by
the topography, eddy energy and enstrophy.

\subsection{Comparison of Mean PVs}

An examination of the similarities and differences
of the 2 PVs

\section{PV Mixing and Unmixing}

A look at models with some from strong pv gradient.

\subsection{Mean Eddy Balances}

Break down of the tendency terms for the mean
dynamics

\subsection{Bounding of E-P Tensor}

A look at how the E-P Tensor is characterised by
the topography, eddy energy and enstrophy.

\subsection{Comparison of Mean PVs}

An examination of the similarities and differences
of the 2 PVs

\section{Strong Flow Models}

A look at models with a strong mean state and hence
a two way mean eddy interaction. E.g. Jets.

\subsection{Mean Eddy Balances}

Break down of the tendency terms for the mean
dynamics

\subsection{Bounding of E-P Tensor}

A look at how the E-P Tensor is characterised by
the topography, eddy energy and enstrophy.

\subsection{Comparison of Mean PVs}

An examination of the similarities and differences
of the 2 PVs

\section{Diagnostic Summary and Eddy-Mean Interaction Predictions}

Niave attempt to parameterise eddies using the 
results from the previous chapters.

\section{Conclusions}










\end{document}