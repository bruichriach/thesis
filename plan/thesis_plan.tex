\documentclass[10pt,a4paper]{article}
\usepackage[latin1]{inputenc}
\usepackage{amsmath}
\usepackage{amsfonts}
\usepackage{amssymb}
\usepackage{graphicx}
\title{Thesis Plan}
\begin{document}

\maketitle


\section{Introduction}

\subsection{Arctic Ocean}

Why the Arctic is important for climate and ocean circulation.

\subsubsection{Thesis Outline}

Outline of hypotheses.

\subsection{Arctic Ocean}

Basically the Arctic part of my CofS
Description of the constituents of the Arctic
and its Dynamics. Topography,
vertical stratification, water masses, circulation.

\subsection{Numerical Models of Arctic}

AOMIP and attempts to have predictive models of
the Ocean. Prominent problems and successes.

\subsection{Mean-Eddy Interaction Theory}

Overview of Residual Theory from GM in 1990 and the
stuff in the late 80's through to M and M 2013.

\subsubsection{GM}

\subsubsection{PV Closure}

I.e. Greatbach

\subsubsection{TWA Equations}

I.e. Young 2012

\subsubsection{Eliason-Palm Tensor}

I.e. M and M 2013

\section{Shallow Water Equations}

My own development of the layered shallow water equations

\subsection{Shallow Water Theory}

Rational for using shallow water equations and
description. I.e. Arctic water masses, 
strong stratification.

Note the use of Rigid Lid approximation in
simulations.

\subsection{Model Discretisation}

C-grid, AB time stepping, Preconditioned CG solver. 

\subsection{Forcing and Dissipation}

The form of wind stress and bottom friction in
shallow water models (e.g. $\tau/h$ and 
$-r\left|\boldsymbol{u}\right|^{2}\boldsymbol{u}/h$). Viscosity:
laplacian, biharmonic as well as Smagorinsky.
Linear, quadratic and n-polynomial bottom friction.

\subsection{Eddy Stresses}

Thickness weighted mean of the shallow water
equation. Form of the momentum and pv eddy stresses.
Thickness weighted PV vs PV derived from thickness 
weighted equations.

Time averaged diagnostics. Explain that this 
requires steady state dynamics. Need configurations
with a unique mean steady state.

\section{Model Descriptions and Mean balances}

A description of the different dynamics that evolve
in the models by using the forcings and 
dissipations described above. Different forcings
such as different wind stresses or regional heating
and cooling. Different dissipation from
linear bottom friction to higher order bottom
friction.

Break down of the tendency terms for the mean
dynamics. A look at how the eddy stresses are 
in balancing the mean states (usually through
balancing the ageostropic component of the
momentum tendency).

\subsection{Zero-Mean Wind Forced Models}

Model where the mean forcing is effectively zero. 
Hence the model is forced by a wind stress which is
absent from the mean system and hence forces the
system by 
invoking a turbulent cascade. Hence Eddy-Mean
interaction is entirely from the eddy to the mean.

\subsection{PV Mixing and Unmixing}

A look at models with some from strong pv gradient.

\subsection{Strong Flow Models}

A look at models with a strong mean state and hence
a two way mean eddy interaction. E.g. Jets in
channels or double gyres. Examination of the
difficulty of defining a mean state.


\section{Comparison of Mean PVs}

An examination of the similarities and differences
of the 2 PVs. A discussion of whether the thickness
weighted and tracer decomposition forms of the
PV eddy stresses can be used interchangeably.

\section{Bounding of E-P Tensor}

A look at how the E-P Tensor is characterised by
the topography, eddy energy and enstrophy.

\section{Diagnostic Summary and Eddy-Mean Interaction Predictions}

Niave attempt to parameterise eddies using the 
results from the previous chapters, depending
on what can be said about how the eddy stress can
be characterised and constrained by the topography,
eddy energy, eddy enstrophy etc. 

Likely will need to carry the time and spatial
evolution of eddy energy and eddy enstrophy and have
some sort of step function to turn on and off the
"GM like" and "holloway like" parts of the parametrisation at appropriate moments.

\section{Conclusions}










\end{document}