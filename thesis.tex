\documentclass[10pt,a4paper]{article}
\usepackage[latin1]{inputenc}
\usepackage{amsmath}
\usepackage{amsfonts}
\usepackage{amssymb}
\usepackage{graphicx}



\usepackage[acronym]{glossaries}
\usepackage[backend=biber,style=bwl-FU,maxbibnames=99]{biblatex}
\addbibresource{bibrefs.bib}

\usepackage{setspace}
%\singlespacing
%\onehalfspacing
\doublespacing
%\setstretch{1.1

%\title{The Importance of Eddy-Topographic Interactions for Arctic
%	Ocean Dynamics and Steps Towards Improving Climate Models}
%\author{Mark Edward Forshaw\\ Department of Physics\\
%University of Oxford\\ E-mail: \url{forshaw@atm.ox.ac.uk}}


\newacronym{amoc}{AMOC}{Antarctic Meridional Overturning Circulation}
\newacronym{bas}{BAS}{British Antarctic Survey}
\newacronym{arp}{ARP}{Arctic Research Program}
\newacronym{ipy}{IPY}{International Polar Year}
\newacronym{awl}{AWL}{Atlantic Water Layer}
\newacronym{aomip}{AOMIP}{Arctic Ocean Model Inter-comparison Project}
\newacronym{pv}{PV}{potential vorticity}
\newacronym{gcm}{GCM}{Global Climate Model}
\newacronym{bso}{BSO}{Barents Sea Opening}

\nocite{cushman2011introduction}
\nocite{vallis2006atmospheric}

\title{Thesis}


\begin{document}

\maketitle


\section{Introduction}

\subsection{Arctic Ocean}


Despite its small size (only $3\%$ of the Earth's surface), the Arctic
Ocean is an important part of the global ocean and has the potential
to affect the world's climate in major ways. Located above
$70\,^{\circ}{\rm N}$, the Arctic is characterised by its seasonally
varying ice cap. This ice cap, along with its counterpart at the South
Pole, is a major contribution to the planet's albedo as well as being
a large reservoir for freshwater. The Arctic Ocean is a key component
in how the polar ice cap interacts with the climate and, in turn, how
the climate affects the ice cap. 
The Ocean itself is a highly stable body of water with strong stratification, 
which means that the water column can be divided into advective water masses.
The Arctic is commonly divided into four different layers:
The Polar Mixed Layer, which is a well mixed layer in
contact with the atmosphere and surface ice; the halocline, which acts as
reservoir of fresh water and penetrates to around ${\sim}200\,{\rm m}$; the \gls{awl},
which is body of warm salty water which originates from the North Atlantic and Nordic 
Sea via Fram Strait and typically sits at around ${\sim}400\,{\rm m}$ deep; and finally, 
the Polar Deep Water, defined by the water masses that aren't
able to communicate freely with the global ocean due to being trapped by the Arctic basin walls.

Because of the strong stratification, the forcing from the ocean surface 
struggles to penetrate deep into the ocean; and so whether forced directly by wind 
or by sea ice stress, the circulation
directly forced from the surface is contained within the Polar Mixed Layer 
and halocline in the upper ${\sim}200\,{\rm m}$ of the Arctic Ocean. 
This surface circulation has been well documented by observations
from cruises and moorings  dropped  from  the  ice  shelf  or  deployed  on 
the  shallower  continental  shelves (\cite{gerdes1997large}, \cite{jones2001circulation}). On the other hand, it's a very different
story for the circulation in the \gls{awl} and lower layers. Observations made
inside the Ocean Basins rarely are able to profile the structure  and
circulation  below 500m  although much progress has been made here in the
past decade. This lack of observations means that  we  still only have  a  sense  of  the circulation 
of the \gls{awl} and the Deep  Waters.  The  observations  we  do 
have  infer stable cyclonic  rim  currents about each of the basins in the
\gls{awl} and it's believed that this trend extends through the Deep Water to the 
ocean floor.  Because of the highly stratified nature of the halocline, the surface forcing 
is likely to have little effect on the deeper ocean, which leaves the other 
two possibilities as the determining features. There has been much discussion
into how the fluxes in and out of the Arctic could influence the circulation
for example by a forcing a balance of \gls{pv} within the region by
the fluxes of \gls{pv} though the boundaries (\cite{yang2005arctic}). 
\begin{itemize}
	\item A brief mention of  Sheldon's budget calculations using boundary fluxes etc.
\end{itemize}
However, the deep water circulation in the Arctic Ocean is more likely determined by 
internal processes, which are poor understood in an Ocean with such sparse observations.

A likely candidate for generating a cyclonic circulation in the Northern hemisphere is
geostrophic turbulence. A number of studies have demonstrated how, in an eddying system, 
sloping topography can generate along-topography flows (\cite{treguier1989topographically}, \cite{adcock2000interactions}, \cite{nost2008asymmetry}). Hence, in the absence of stronger
forcing, this effect is likely to become the dominant feature. The obvious question that then
comes to mind is how turbulent is the Arctic Ocean? Over recent years a large amount  of 
effort has been made to observe and catalogue eddies in the Arctic
(\cite{zhao2014characterizing}) and has demonstrated their prevalence in the Arctic Ocean.
Whilst these studies have frequently been focused on the halocline and the surface dynamics
there is good reason to believe that eddies also penetrate deep into the Arctic 
(\cite{woodgate2001arctic}).

\subsection{Mean-Eddy Interaction Theory}

 Obviously, whilst observations play a vital role in understanding the Arctic
 Ocean and its role in the global ocean, it is limited by the sparsely available
 observations that only usefully reach back a handful of decades into the past.
 To be able to test possible scenarios and make future predictions it is vital
 that we have reliable simulations of the Arctic Ocean (\cite{proshutinsky2008toward}). 
 However, due to limitations
 in modern computing power and poor understanding of the unresolved processes 
 in the ocean this is, more often than not, far from reality.
 By attempting  to understand the physics behind these unresolved processes 
 and developing parameterisations that mimic their effect on the resolved ocean
 the value of computational models can be improved without dramatically increasing the
 computing power required.  Due to these computational limits a practical \gls{gcm} is 
 limited to a  resolution of the order of tens of kilometres, this means that with a 
 resolution  cut off at around $10 \textendash 100 \, \rm km$, 
 \glspl{gcm} are unable to fully resolve a large portion of the mesoscale turbulence that is
 observed in the physical ocean.
 
 The turbulent, chaotic nature of the ocean is caused by the non-linearity of its 
 governing equations. Because of this,  the discretisation of the equations generates cross-correlation, or eddy, terms when applying the 
 averaging operator,
 \begin{equation}
 \overline{\phi\psi} = \overline{\phi}\,\overline{\psi} + 
 \overline{\phi^{\prime}\psi^{\prime}},
 \label{non-lin average}
 \end{equation}
 where the prime denotes the residual, ${\phi^{\prime} = \phi - \overline{\phi}}$,
 and the ${\overline{\phi}}$ is the average of the variable ${\phi}$, used in
 the discretisation and satisfying
 the expected properties of an averaging operator.
 The discretised system has no knowledge of the residual terms and hence
 the contributions to the full system by these eddy terms are
 ignored in the discretised system. It is, therefore, this ``sub-grid scale process"
 that is missing from the model dynamics and so it is common practise to try and
 `close' the system by including a parameterisation of the eddy
 term in place of the term itself, which is usually dependent on the averaged 
 variables and some physical assumptions.


\end{document}