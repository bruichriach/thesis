\documentclass[10pt,a4paper]{article}
\usepackage[latin1]{inputenc}
\usepackage{amsmath}
\usepackage{amsfonts}
\usepackage{amssymb}
\usepackage{graphicx}
\title{Plan}
\begin{document}

\maketitle


\section{Introduction}

\subsection{Arctic Ocean}

An, albeit relatively brief, description of the Arctic Ocean. This will focus on:
\begin{itemize}
\item Defining the Arctic water masses based upon the vertical profile and stratification.
\item Observed circulation in the Arctic, particularly in the Atlantic Water Layer.
\item A brief mention of  Sheldon's budget calculations using boundary fluxes etc.
\end{itemize}

An area that I've not touched on yet is eddies in the Arctic, mostly because the literature
is still very sparse. Papers like Zhao 2014 give some idea of their prevalence, at least in the
halocline. The Woodgate paper is also useful for this.

\subsection{Mean-Eddy Interaction Theory}

Overview of the development of eddy parameterisations over the years along
the lines of what was briefly described in the Transfer of Status report. 
A discussion of how
modern climate models implement eddy parameterisations
as well as the positives and limitation of current 
eddy parameterisation techniques. 
This is more specifically a conversation on the development
of the Gent and McWilliams parameterisation. 
How it is implemented, what it does and how it demonstrates itself in the Arctic.
\begin{equation}
  \frac{\partial\tau}{\partial t} + (\boldsymbol{u}\cdot\boldsymbol{\nabla})\tau  = -\boldsymbol{\nabla}(\kappa\sigma
  \boldsymbol{\nabla}\tau)/\sigma,
\end{equation}
\begin{itemize}
\item MITgcm uses GM as a tracer mixer (see GMREDI)
\end{itemize}

A discussion alternative approaches to parametrising eddies,
such as potential vorticity closure  and 
the Neptune Effect, and how this
leads onto the Eliason-Palm Tensor.
Schematic c.f. A\&M00 but basin instead of sea mount.


\subsection{Numerical Models of Arctic}

AOMIP and attempts to have predictive models of the Arctic Ocean. This includes attempts
to implement the Neptune effect or other parameterisations such as MEP as well as
the increased success of using higher resolution eddy permitting models.

Somewhere amongst either this section or the previous I'd like to mention Yang 2005, the 
PV balance explanation for a cyclonic Atlantic Water Layer.

\section{Developement of Shallow Water Equations}

The theoretical advancement made as part of this thesis, as well as the technical development 
needed to test hypotheses. The development of the 
multi-layered  shallow water approximation case of the 
Eliason-Palm eddy closure theory.

\subsection{Shallow Water Theory}

Rational for using shallow water equations and
description. I.e. Arctic water masses, 
strong stratification. Development of the eddy closure theory in the shallow water case
including examining certain nuances the approximation includes, such
as the description of physical quantities, like
potential and kinetic energy, on the discontinuous 
jump between layers.

Note the use of Rigid Lid approximation in
simulations.

\subsection{Model Discretisation}

Continuous model to discontinous discrete model.
Technical description of the models. 
C-grid, AB time stepping, Preconditioned CG solver. 
Parallelisation.
Forcing and Dissipation used in the model.
The form of wind stress and bottom friction in
shallow water models (e.g. $\boldsymbol{\tau}/h$ and 
$-r\left|\boldsymbol{u}\right|^{2}\boldsymbol{u}/h$). Viscosity:
laplacian, biharmonic as well as Smagorinsky.
Linear, quadratic and n-polynomial bottom friction.

Example model output.

\subsection{Eddy Stresses}

Thickness weighted mean of the shallow water
equation. Form of the momentum and pv eddy stresses.
Thickness weighted PV vs PV derived from thickness 
weighted equations.

Eddy energy and eddy enstrophy. State equations with 
possible application in terms of carrying them in the system.

Time averaged diagnostics. Explain that this 
requires steady state dynamics. Need configurations
with a unique mean steady state.

\section{Analysis of a hierarchy of models}

A description of the different dynamics that evolve
in the models by using the forcings and 
dissipations described above. Different forcings
such as different wind stresses or regional heating
and cooling. Different dissipation from
linear bottom friction to higher order bottom
friction.

Break down of the tendency terms for the mean
dynamics. A look at how the eddy stresses are 
in balancing the mean states (usually through
balancing the ageostropic component of the
momentum tendency).

A closer look at the eddy stress, their bounds 
and the available eddy budgets.

\subsection{Zero-Mean Wind Forced Models}

Model where the mean forcing is effectively zero. 
Hence the model is forced by a wind stress which is
absent from the mean system and hence forces the
system by 
invoking a turbulent cascade. Hence Eddy-Mean
interaction is entirely from the eddy to the mean.

\subsection{Comparison of Mean PVs}

An examination of the similarities and differences
of the two definitions of potential vorticity. 
A discussion of whether the thickness
weighted and tracer decomposition forms of the
PV eddy stresses can be used interchangeably.
This will be done by examining models with
different structures of potential vorticity.

\subsection{Strong Flow Models}

more interesting situations, such as, models
with a strong mean state, which hence
generate there own eddy fields. E.g. Jets in
channels or double gyres. Hopefully allowing 
for the investigation of configurations where 
Eddy Enstrophy is the constraining bound.


\section{Diagnostic Summary and Eddy-Mean Interaction Predictions}

Na\"{\i}ve attempt to parameterise eddies using the 
results from the previous chapters, depending
on what can be said about how the eddy stress can
be characterised and constrained by the topography,
eddy energy, eddy enstrophy etc. 

Likely will need to carry the time and spatial
evolution of eddy energy and eddy enstrophy and have
some sort of step function to turn on and off the
"GM like" and "holloway like" parts of the parametrisation at appropriate moments.


\section{conclusions and discussions}

Summary of results.
Future work (keep this as unexplored possible ideas for now)

\end{document}