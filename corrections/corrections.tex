\documentclass[10pt,a4paper]{article}
\usepackage[latin1]{inputenc}
\usepackage{amsmath}
\usepackage{amsfonts}
\usepackage{amssymb}
\usepackage{graphicx}
\usepackage{color, soul}
\begin{document}
	
	
	Abstract:
	\begin{itemize}
		\item paragraph 1: removed final sentence of paragraph.
		
		\item paragraph 3, lines 1-6: added two sentences highlighting energy bound and PV results.  
	\end{itemize}
	
	Acknowledgements:
	\begin{itemize}
		\item Acknowledgement page added. \hl{not completed}
	\end{itemize}
	
	Chapter 1, Introduction:
	\begin{itemize}
		\item page 3, paragraph 1, line 9: `advective water masses' changed to `well
		defined water masses'
		\item page 3, paragraph 1, line 11: `and penetrates'  changed to `penetrating down'
		\item page 6, paragraph 2, line 9:  `unconvincing' changed to `unclear'
		\item page 6, paragraph 2, line 12:  `most likely' changed to `may be'
		\item page 6, paragraph 2, line 12:  `poor understood' changed to `poorly understood'
		\item page 7, figure 1.1.3: Figure rotated to allow it to be larger.
		\item page 8, figure 1.1.4, caption: `arrays; Demonstrates' changed to `arrays. These demonstrate'
		\item page 9, paragraph 1: equation removed.
		\item page 9, last two lines but one:
		`a quantity with dimensions of length which represents the ratio of sensitivities to the total energy and enstrophy in the system.'
		changed to
		`a quantity which represents the ratio of 
		sensitivities to the total energy and enstrophy in the system and has
		dimensions which are the inverse square of length.'
		\item page 14, paragraph 2, line 5: `in an irresponsible manner' changed to `in such a manner that does not respect the conserved properties of the system'
	\end{itemize}
	Chapter 2:
	\begin{itemize}
		\item page 16, line 5: added sentence: `The operator could be an averaging over space, time, a large ensemble or a combination of these.'
		\item section 2.1, paragraph 1, last sentence: `A second problem with having such a strong horizontal diffusion is
		that it diffuses or mixes tracers horizontally, when applied as
		an eddy parameterisation for tracer mixing.' changed to `This is usually represented by a horizontal diffusion, hence
		a second problem is
		that such a strong horizontal diffusion diffuses or mixes tracers excessively.'
		\item page 17, line after equation 2.1.1: `defined as' changed to `the negative of what is known as'.
		\item page 17, 2 lines after equation 2.1.1: added `and $\boldsymbol{\nabla}_{\rho}$ is the horizontal divergence operator as
		presented in Young (2012) defined in terms of $\rho$ rather than the buoyancy, $b$.'
		\item page 17, equations 2.1.1, 2.1.2 and 2.1.3: changed $\boldsymbol{\nabla}$ to $\boldsymbol{\nabla}_{\rho}$.
	\end{itemize}
	
	
	
	
\end{document}